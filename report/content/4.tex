\chapter{Findings and discussion}
\section{Model usage}
Given the regression model, it can be applied to example data to examine its behavior.
\subsection{Assessing expectations}
Intuitive assumptions draw you to the conclusion, that the vehicle's mileage has an inverse relationship to the predicted advertisement price.
To evaluate if the model also follows this behavior, it was applied to manually created data differentiating only by mileage.
\begin{table}[H]
    \begin{adjustbox}{width={\textwidth}}
        \begin{tabular}{|c|c|c|c|c|c|c|c|}
            \hline
            Registration & \textbf{Mileage} & Horse- & Width & Length & Average & Top speed & \textbf{Predicted price} \\[-1ex]
            year         & \textbf{(mi)}    & power  & (mm)  & (mm)   & mpg     & (mph)     & \textbf{(£)}             \\ \hline
            2017         & \textbf{60000}   & 135    & 2027  & 4284   & 49      & 116       & \textbf{16157}           \\\hline
            2017         & \textbf{130000}  & 135    & 2027  & 4284   & 49      & 116       & \textbf{13828}           \\\hline
        \end{tabular}
    \end{adjustbox}
    \caption{Influence of mileage on recently registered cars}
    \label{mileage_data_new_car}
\end{table}
As it is evident in \autoref{mileage_data_new_car}, a higher mileage in fact reduces the predicted price.
However, for an over 2-fold increase in miles, the depreciation is with 14.4 \% not as high as initially expected.
\par
While comparing two otherwise identical cars, it should be noted that the other values still contribute significantly to the
result.
\newline
In particular the registration year, which, as shown before, strongly correlates with the target variable has an effect on the limited
influence of the mileage here.
Given the data set's sampling of data up to 2017, both of the \ac{vw} Golfs in \autoref{mileage_data_new_car} have been first registered only one year ago, so the base price is measurably higher.
If you apply the same example to cars registered in 2010, which therefore have been running for seven years, the influence of the mileage grows.
\begin{table}[H]
    \begin{adjustbox}{width={\textwidth}}
        \begin{tabular}{|c|c|c|c|c|c|c|c|}
            \hline
            \textbf{Registration} & \textbf{Mileage} & Horse- & Width & Length & Average & Top speed & \textbf{Predicted price} \\[-1ex]
            \textbf{year}         & \textbf{(mi)}    & power  & (mm)  & (mm)   & mpg     & (mph)     & \textbf{(£)}             \\ \hline
            \textbf{2010}         & \textbf{60000}   & 135    & 2027  & 4284   & 49      & 116       & \textbf{11122}           \\\hline
            \textbf{2010}         & \textbf{130000}  & 135    & 2027  & 4284   & 49      & 116       & \textbf{8793}            \\\hline
        \end{tabular}
    \end{adjustbox}
    \caption{Influence of mileage on older cars}
    \label{mileage_data_old_car}
\end{table}
As shown in \autoref{mileage_data_old_car}, the gap between the two otherwise identical vehicles has widened to 20.9 \%, an increase by 45.3 \%.
Potential reasons for the strong correlation between year and target value will be investigated further in \autoref{year_to_price_correlation}.
Nevertheless, for that small subset of data, the efficacy of the model is evident.
\subsection{Usage of example data}
However, this small example is not cohesive enough to demonstrate the ability to solve the business problem.
To apply the model to a day-to-day use case as it regularly appears in a dealership, it was used to predict the advertisement price
of four cars as they could be in its yard.
\begin{table}[H]
    \begin{adjustbox}{width={\textwidth}}
        \begin{tabular}{|c|c|c|c|c|c|c|c|}
            \hline
            Registration & Mileage & Horse- & Width & Length & Average & Top speed & \textbf{Predicted price} \\[-1ex]
            year         & (mi)    & power  & (mm)  & (mm)   & mpg     & (mph)     & \textbf{(£)}             \\ \hline
            2014         & 180000  & 110    & 1799  & 4204   & 45      & 110       & \textbf{5601}            \\\hline
            2016         & 150000  & 120    & 2027  & 4255   & 48      & 112       & \textbf{12130}           \\\hline
            2018         & 80000   & 130    & 2027  & 4255   & 50      & 115       & \textbf{16136}           \\\hline
            2015         & 190000  & 115    & 1799  & 4204   & 44      & 108       & \textbf{5819}            \\ \hline
        \end{tabular}
    \end{adjustbox}
    \caption{Assessing model for business use cases}
    \label{predicted_price_realworld_data}
\end{table}
\section{Findings}
\subsection{In line with presumptions}
\subsubsection{Mileage $\leftrightarrow$ Price}
\subsubsection{Engine size $\leftrightarrow$ Price}

\subsection{Outliers}
\subsubsection{Engine size $\leftrightarrow$ Price}
\subsubsection{Width $\leftrightarrow$ Price}
\subsubsection{Year $\leftrightarrow$ Price} \label{year_to_price_correlation}
TODO include example of higher influence of mileage the older the car is
