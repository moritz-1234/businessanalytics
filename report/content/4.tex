\chapter{Findings and discussion}
\section{Model usage}\label{model_usage_section}
Given the regression model, it can be applied to example data to examine its behavior.
\subsection{Assessing expectations}
Intuitive assumptions draw you to the conclusion, that the vehicle's mileage has an inverse relationship to the predicted advertisement price.
To evaluate if the model also follows this behavior, it was applied to manually created data differentiating only by mileage.
\begin{table}[H]
    \begin{adjustbox}{width={\textwidth}}
        \begin{tabular}{|c|c|c|c|c|c|c|c|}
            \hline
            Registration & \textbf{Mileage} & Horse- & Width & Length & Average & Top speed & \textbf{Predicted price} \\[-1ex]
            year         & \textbf{(mi)}    & power  & (mm)  & (mm)   & mpg     & (mph)     & \textbf{(£)}             \\ \hline
            2017         & \textbf{60000}   & 135    & 2027  & 4284   & 49      & 116       & \textbf{16157}           \\\hline
            2017         & \textbf{130000}  & 135    & 2027  & 4284   & 49      & 116       & \textbf{13828}           \\\hline
        \end{tabular}
    \end{adjustbox}
    \caption{Influence of mileage on recently registered cars}
    \label{mileage_data_new_car}
\end{table}
As it is evident in \autoref{mileage_data_new_car}, a higher mileage in fact reduces the predicted price.
However, for an over 2-fold increase in miles, the depreciation is with 14.4 \% not as high as initially expected.
\par
While comparing two otherwise identical cars, it should be noted that the other values still contribute significantly to the
result.
\newline
In particular the registration year, which, as shown before, strongly correlates with the target variable, has an effect on the limited
influence of the mileage here.
Given the data set's sampling of data up to 2017, both of the \ac{vw} Golfs in \autoref{mileage_data_new_car} have been first registered very recently, so the base price is measurably higher.
If you apply the same example to cars registered in 2010, which therefore have been running for seven years, the influence of the mileage grows.
\begin{table}[H]
    \begin{adjustbox}{width={\textwidth}}
        \begin{tabular}{|c|c|c|c|c|c|c|c|}
            \hline
            \textbf{Registration} & \textbf{Mileage} & Horse- & Width & Length & Average & Top speed & \textbf{Predicted price} \\[-1ex]
            \textbf{year}         & \textbf{(mi)}    & power  & (mm)  & (mm)   & mpg     & (mph)     & \textbf{(£)}             \\ \hline
            \textbf{2010}         & \textbf{60000}   & 135    & 2027  & 4284   & 49      & 116       & \textbf{11122}           \\\hline
            \textbf{2010}         & \textbf{130000}  & 135    & 2027  & 4284   & 49      & 116       & \textbf{8793}            \\\hline
        \end{tabular}
    \end{adjustbox}
    \caption{Influence of mileage on older cars}
    \label{mileage_data_old_car}
\end{table}
As shown in \autoref{mileage_data_old_car}, the gap between the two otherwise identical vehicles has widened to 20.9 \%, an increase by 45.3 \%.
Potential reasons for the strong correlation between year and target value will be investigated further in \autoref{year_to_price_correlation}.
Nevertheless, for that small subset of data, the efficacy of the model is evident.
\subsection{Usage of example data}
However, this small example is not cohesive enough to demonstrate the ability to solve the aforementioned business problem.
To apply the model to a day-to-day use case as it regularly appears in a dealership, it was used to predict the advertisement price
of four automobiles as they could be in its yard.
\begin{table}[H]
    \begin{adjustbox}{width={\textwidth}}
        \begin{tabular}{|c|c|c|c|c|c|c|c|}
            \hline
            Registration & Mileage & Horse- & Width & Length & Average & Top speed & \textbf{Predicted price} \\[-1ex]
            year         & (mi)    & power  & (mm)  & (mm)   & mpg     & (mph)     & \textbf{(£)}             \\ \hline
            2014         & 180000  & 110    & 1799  & 4204   & 45      & 110       & \textbf{5601}            \\\hline
            2016         & 150000  & 120    & 2027  & 4255   & 48      & 112       & \textbf{12130}           \\\hline
            2018         & 80000   & 130    & 2027  & 4255   & 50      & 115       & \textbf{16136}           \\\hline
            2015         & 190000  & 115    & 1799  & 4204   & 44      & 108       & \textbf{5819}            \\ \hline
        \end{tabular}
    \end{adjustbox}
    \caption{Assessing model for business use cases}
    \label{predicted_price_realworld_data}
\end{table}
\autoref{predicted_price_realworld_data} illustrates that the model is able to predict an appropriate advertisement price
for a \ac{vw} Golf. As the model's performance with $R^2 = 0.927$ is very good, the dealer can rely on it, receiving a data driven
estimate for a competitive price, thus not having to rely solely on human estimate. 
\section{Findings}
In \autoref{model_usage_section} the application of the model to exemplary vehicles has been demonstrated. %TODO add reference to simons part on test data
Here, some trends that could have already been anticipated after the correlation analysis, have emerged more clearly.  
\newline
These trends that have been found regarding the model and the underlying data can be divided into two groups, the predicted and the unexpected.
\subsection{In line with presumptions}
In this first part, the findings which can be considered logical or self-evident will be discussed. 
\subsubsection{Mileage $\leftrightarrow$ Price}
Mileage as an indicator can be seen as an indicator for wear of the vehicle. It directly implies more miles driven,
yet indirectly affects other markers as well. It may mean doors have been opened and closed more often, 
electronics have been running for longer, the paint has suffered more damage from washing. 
Apart from that, maintenance parts are closer to their end of life and need to be replaced sooner,
which costs customers money and time. 
Therefore, the negative correlation of -0.304 between mileage and price is behaving as expected, albeit 
overshadowed by the influence of the registration year. This phenomenon will be explored further in \autoref{year_to_price_correlation}.
\subsubsection{Horsepower $\leftrightarrow$ Price}
On of the essential specifications of a vehicle is its power, measured in horsepower.
It affects the maximum acceleration, the top speed, the fuel consumption and other key indicators of a car's capabilities.
The improved performance is a reason, why customers are willing to spend extra for a car with more horsepower. 
As with new vehicles, the direct relationship between horsepower and price also manifests in the pre-owned market, %TODO quelle dafür?
with a correlation of 0.465.
\subsection{Outliers}
Nevertheless, there are also some outliers which ought to be investigated in more detail. 
\subsubsection{Engine size $\leftrightarrow$ Price}
The engine displacement, informally also referred to as engine size, describes the volume of air and fuel inside an engine's pistons \autocite{EngineDisplacement2024}
and is measured in liters in the data set.
By definition, it is also related to its power output, in particular in older vehicles.
\par
Given that, the intuitive prediction is that it will positively correlate with the final price. 
However, the analysis has shown that there the correlation is not significant, with -0.063 approaching 0.
There are two main reasons for this. 
\par
Exploring the correlation of the engine size with other parameters, the lack of effect on the final price can be explained by looking at
three key indicators:
\begin{itemize}
    \item \textbf{Top speed: }
    With 0.552, there is a significant effect of the engine size on the top speed, increasing the predicted value of the Golf.
    \item \textbf{Registration year: }
    There is a measurable trend that for newer cars, the engines become smaller, notable by a coefficient of -0.295.
    Due to the strong effect of the registration year on the predicted price, the higher engine sizes negatively influence the target value,
    compensating the aforementioned effect of top speed.
    \item \textbf{Horsepower: }
    In newer cars, horsepower is not as limited by engine size as in the past \autocite{WhatEngineDisplacement}. 
    Looking at the correlation value of -0.076, this effect is also resembled in VW Golfs. Thus, one of the deciding factors determining the end price
    is not related to the engine size.
\end{itemize}
To conclude, the two measurable effects even each other out and for the horsepower, where correlation might be present, there is not any, rendering the 
engine size irrelevant for our analysis. %TODO verbessern
\subsubsection{Width $\leftrightarrow$ Price}
At the first glance, a very strong tie between the width of a Golf and its advertisement price is not clear.
\subsubsection{Year $\leftrightarrow$ Price} \label{year_to_price_correlation}
TODO include example of higher influence of mileage the older the car is
