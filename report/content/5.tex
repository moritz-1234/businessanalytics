\chapter{Conclusion}
\section{Further questions}


\subsubsection{Lack of sales data regarding advertisement}
While the given dataset included cohesive data regarding the advertisement price of cars in the used vehicle market,
there is no indicator given, whether the car was actually sold at the price that it has been advertised for. 
\newline
Nevertheless, while you can expect a few dealerships to over- / undershoot their prices,
considering the scale of the dataset for the overall market you can expect that effect to even out.
However, evaluating whether there is a trend that used cars are systematically under- / overpriced in advertisements
is only possible if you compare the given advertisement data set to data containing real sales. %TODO verbessern
\subsubsection{Inter-model comparison of findings}

\subsubsection{Assess value depreciation}
One possible further research topic is the comparison of a car's new price
to its future advertisement prices. For each model in the used-car advertisement dataset, there is a corresponding data point
that contains, among others, the \ac{msrp}. 
\newline
Given that information, the following questions could be analyzed: 
\begin{itemize}
\item Which model retains the most value compared to its \ac{msrp}?
\item Given five years of use, which car's price decreased the most?
\item Is there a correlation between value depreciation and the manufacturer of the car? 
\end{itemize}
This information can be useful to customers considering the purchase of a new car in order to assess its potential resale value in the future.


\section{Résumé}
   