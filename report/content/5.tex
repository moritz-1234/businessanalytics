\chapter{Conclusion}
\section{Résumé}

The conducted analysis provides a data-driven approach to the business problem of a competitive advertisement price of a given \ac{vw} Golf.
After preparing the raw data, key influences have been separated using Pearson-correlation and used to train a linear regression model consisting of the parameters:
\begin{itemize}
    \item Registration year
    \item Mileage (mi)
    \item Horsepower
    \item Width (mm)
    \item Length (mm)
    \item Average mpg
    \item Top speed (mph)
\end{itemize}
It accurately predicts a suitable advertisement price with $R^2 = 0.927$, solving the business problem thereby.
Revealing the most relevant factors affecting the price solves research question 1, while the model and its prediction capabilities solve research question 2 by enabling a price prediction for a \ac{vw} Golf.
The end results mostly align with logical assumptions, for instance that newer cars sell for higher prices, yet also reveal
surprising behaviors such as the significant effect of the Golf's width.
To summarize, they provide further insight by revealing each factor's contribution to the vehicle's advertisement price and investigating their cause.
\section{Limitations of the analysis}
These results, while showing conclusive behavior among the analyzed aspects, are mainly constrained by two factors.
\subsection{Recency of the data}
Firstly, the training data has been collected between 2016 and 2018, thus even the most recent data
is now over 5 years old. Given recent events such as the steep increase of inflation
and the COVID-19 pandemic, predictions by the model might not accurately reflect current trends.

\subsection{Lack of sales data regarding advertisement}
Additionally, while the given dataset includes conclusive data regarding the advertisement price of cars in the used vehicle market,
there is no indicator given, whether the car was actually sold at the price that it has been advertised for. 
\newline
Nevertheless, while one may anticipate a few dealerships to over or underestimate their prices,
considering the scale of the dataset, that effect is expected to level out for the overall market.
However, evaluating whether there is a trend that pre-owned cars are systematically under- / overpriced in commercials
is only possible if you compare the given data set to real sales information.

\section{Further research topics}
However, given the amount of available data in the data set, more potential research questions may also be examined in the future. 

\subsection{Inter-model comparison of findings}
The analysis is currently only valid for the small subset of the data including the \ac{vw} Golf. 
To elevate generalizability to the whole used vehicle market and to assess potential disparities as well as similarities among
car models, expanding the scope of the study to the entirety of available data is recommended.  
\subsection{Assess value depreciation}
After evaluating inter-model differences, a possible further research topic is the comparison of each car's new price
to its future advertisement prices. For each model in the advertisement dataset, there is a corresponding data point
in the "basic information" table that contains, among others, the \ac{msrp}. 
\newline
With that information, the following questions could be analyzed: 
\begin{itemize}
\item Which model retains the most value compared to its \ac{msrp}?
\item Given five years of use, which car's price decreased the most?
\item Is there a correlation between value depreciation and the manufacturer of the car?
\end{itemize}
This information can be useful for customers considering the purchase of a new car in order to assess its potential resale value in the future.

%TODO end sentence
%TODO: Answer research questions
