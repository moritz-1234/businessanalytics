\chapter{Task}
\section {Robot station}
Composition of the real robotic station with an explanation of the function of each part.
\section {Kinematic scheme}
A kinematic scheme of the robot.

\section {Rapid programs}
Your executable RAPID modules with an explanation of corrections you made in the module Description section.
\section{Conclusion}
Conclusions on the program implementation and modification
\chapter{Solution}
\section {Robot station}

In \autoref{robot_station} three parts of the station are recognizable.
\par
\textbf{A:} ABB IRB 14000 robot\newline
\textbf{B:} FlexPendant robot controller\newline
\textbf{C:} Lab computer connected to FlexPendant. From there, the program is transferred to the robot.
\begin{figure}[H]
    \centering
    \includegraphics[width=0.9\textwidth]{images/station.jpg}
    \caption{Robot station}
    \label {robot_station}
\end{figure}


\section {Kinematic scheme}
\begin{figure}[H]
    \centering
    \includegraphics[width=0.9\textwidth]{diagrams/kinematic_diagram_b.png}
    \caption{Kinematic scheme}
    \label {kinematic_diagram}
\end{figure}


\section {Rapid programs}
\subsection{Lab 1}
\lstinputlisting{rapid/lab1_l_new.mod}
\lstinputlisting{rapid/lab1_r_new.mod}
\subsection{Lab 2}
\lstinputlisting{rapid/lab2_l_new.mod}

\section{Conclusion}
To adapt the simulated programs to the real robot, the main change that has been made was ensuring a similar starting position between runs.
For that, the robot is always returned to its home position after each program.
\par
After that, the simulated programs had to be tested with slow speed to make sure the gripper does not hit the table and the preconfigured targets were correct.
\newline
Finally, it was a very rewarding experience to see the simulated programs working on the real robot. 

